\documentclass{article}
\usepackage{listings}

\title{Projektdokumentation — Raspberry Pi}
\author{Luca Heidemann}
\date{Mai 2025}

\begin{document}

\maketitle
\newpage

\section{Vorbereitung}
Auf dem Raspberry Pi müssen vorher verschiedene Dinge installiert, aktiviert und eingerichtet werden, damit dieses Projekt möglich ist.
\subsection{Verbindung mit dem WLAN herstellen}
Da die Authentifizierung im WLAN in der Schule über Benutzername und Passwort läuft, sind hier zwei Befehle erforderlich, damit die Verbindung erfolgreich hergestellt wird.
\begin{verbatim}
    sudo nmcli connection add \
        type wifi \
        con-name "DBB-BYOD" \
        ifname wlan0 \
        ssid "DBB-BYOD" \
        wifi-sec.key-mgmt wpa-eap \
        802-1x.eap ttls \
        802-1x.phase2-auth mschapv2 \
        802-1x.identity "heidemann.luca.14462"

    sudo nmcli device wifi connect DBB-BYOD --ask
\end{verbatim}
Nach dem zweiten Befehl erscheint die Abfrage nach einem Passwort. Hier muss dasselbe Passwort wie zur Anmeldung an den PCs in der Schule benutzt werden.
\subsection{Installation PostgreSQL}
Auf dem Raspberry Pi muss eine PostgreSQL-Datenbank laufen. Diese kann mit dem folgenden Befehl installiert werden:
\begin{verbatim}
    sudo apt-get install postgresql
\end{verbatim}
Außerdem wird der PostgreSQL-Treiber für Python benötigt.
\begin{verbatim}
    sudo apt-get install python3-psycopg2
\end{verbatim}
\subsection{Erstellen der benötigten Tabellen}
Mit dem folgenden Befehl kann man eine Verbindung zur Datenbank auf der Kommandozeile herstellen:
\begin{verbatim}
    sudo -u postgres psql
\end{verbatim}
Anschließend kann man sich für sein Projekt eine Datenbank und einen Benutzer erstellen. In meinem Fall heißen Datenbank und Benutzer 'klimatrack'.
\begin{verbatim}
    create database 'klimatrack';
    create user 'klimatrack' with encrypted password 'sec';
    alter database klimatrack owner to klimatrack;
\end{verbatim}
Nun kann man direkt eine Verbindung zur erstellten Datenbank herstellen:
\begin{verbatim}
    sudo -u postgres psql -d klimatrack
\end{verbatim}
Nun erstellt man die benötigte Tabellen, um die Temperaturdaten zu speichern.
\begin{verbatim}
    create table temperatur_aufzeichnungen (
        aufzeichnung_id uuid not null primary key,
        aufzeichnung_wert real not null,
        aufzeichnung_zeitpunkt timestamptz not null
    );
\end{verbatim}
Abschließend gibt man dem erstellten Benutzer die vollen Berechtigungen für die Tabelle:
\begin{verbatim}
    grant all on all tables in schema public to klimatrack;
\end{verbatim}
\subsection{DS18B20 einrichten}
Damit der Temperatursensor am Raspberry Pi funktioniert, muss folgender Eintrag am Ende der Datei '/boot/firmware/config.txt' hinzugefügt werden:
\begin{verbatim}
    dtoverlay=w1-gpio
\end{verbatim}
Anschließend muss der Raspberry Pi neu gestartet werden.
\subsection{Installation Matplotlib zur Diagrammerstellung}
Damit das Skript in dem E-Mails Diagramme erstellen und einbinden kann, muss folgender Befehl ausgeführt werden:
\begin{verbatim}
    sudo apt-get install python3-matplotlib
\end{verbatim}
\subsection{Installation Flask zur Web API}
Außerdem müssen folgende Befehle ausgeführt werden, damit die (bescheidene) Web API funktioniert:
\begin{verbatim}
    sudo apt-get install python3-flask python3-flask-cors
\end{verbatim}
\subsection{I2C einrichten}
Damit der QR-Code auf dem Display angezeigt werden kann, muss zuerst die I2C-Schnittstelle über folgenden Befehl aktiviert werden:
\begin{verbatim}
    sudo apt-get install i2c-tools
    sudo raspi-config
\end{verbatim}
Dort navigiert man nach '3 Interface Options', 'I5 I2C' und bestätigt den Dialog mit 'Yes'. Anschließend muss der Raspberry Pi erneut neu gestartet werden.
Nun mit folgendem Befehl prüfen, ob das Display erkannt wird:
\begin{verbatim}
    $ sudo i2cdetect -y 1
     0  1  2  3  4  5  6  7  8  9  a  b  c  d  e  f
    00:                         -- -- -- -- -- -- -- -- 
    10: -- -- -- -- -- -- -- -- -- -- -- -- -- -- -- -- 
    20: -- -- -- -- -- -- -- -- -- -- -- -- -- -- -- -- 
    30: -- -- -- -- -- -- -- -- -- -- -- -- 3c -- -- -- 
    40: -- -- -- -- -- -- -- -- -- -- -- -- -- -- -- -- 
    50: -- -- -- -- -- -- -- -- -- -- -- -- -- -- -- -- 
    60: -- -- -- -- -- -- -- -- -- -- -- -- -- -- -- -- 
    70: -- -- -- -- -- -- -- --
\end{verbatim}
\subsection{SSD1306 einrichten}
Damit man nun in Python mit dem Display interagieren kann, ist es notwendig, foglende Befehle auszuführen:
\begin{verbatim}
    sudo apt-get install python3-luma.oled
\end{verbatim}
\subsection{Bibliotheken für QR-Code-Generierung installieren}
Folgende Befehle müssen ausgeführt werden, damit die Erzeugung des QR-Codes für das Display einwandfrei funktioniert:
\begin{verbatim}
    sudo apt-get install python3-netifaces python3-pil python3-qrcode
\end{verbatim}
\section{Auslesen der Werte und Füllen der Datenbank}
Verantwortlich für das Auslesen der Werte des Temperatursensors ist die Datei 'write\_temperature\_to\_database.py'. Diese liest die Werte des Temperatursensors alle fünf Sekunden aus und schreibt diese in die Datenbank.
Das passiert auch nach einem Neustart automatisch durch folgenden Eintrag im Crontab, welcher mit folgendem Befehl bearbeitet werden kann:
\begin{verbatim}
    crontab -e
\end{verbatim}
Dort muss am Ende der Datei folgender Eintrag hinzugefügt werden:
\begin{verbatim}
    @reboot ~/klimatrack/write_temperature_to_database.py
\end{verbatim}
\section{Verschicken der Werte per E-Mail bei Tastendruck}
Verantwortlich für das Versenden der E-Mails bei Tastendruck ist die Datei 'send\_email\_on\_button\_click.py'. Diese sendet, sobald der Knopf gedrückt wurde, eine E-Mail mit den Statistiken der letzten 24 Stunden.
Damit auch dies nach einem Neustart wieder funktioniert, muss auch hierfür ein neuer Eintrag zum Crontab hinzugefügt werden:
\begin{verbatim}
    @reboot ~/klimatrack/send_email_on_button_click.py
\end{verbatim}
Wichtig hierbei ist, dass frühstens 60 Sekunden nach der letzten E-Mail eine neue verschickt werden kann. Das hat den Grund, dass sonst möglicherweise mit einem Tastendruck mehrere E-Mails verschickt werden würden.
\section{Web API}
Für die Web API auf Port 8080 ist die Datei `simple\_web\_api\_flask.py` verantwortlich. Natürlich wird auch hierfür wieder ein Eintrag zum Crontab hinzugefügt:
\begin{verbatim}
    @reboot ~/klimatrack/simple_web_api_flask.py
\end{verbatim}
\subsection{GET /}
Dieser Pfad zeigt ein simples Frontend (ähnlich der E-Mail) an, auf welchem man die aktuellen Temperaturdaten anzeigen kann.
\subsection{GET /api/v1/temperatur\_aufzeichnungen}
Dieser Pfad gibt die Temperaturdaten der letzten 24 Stunden als JSON aus.
\section{QR-Code als einfacher Verweis zum Dashboard}
Damit der Aufruf der Web API so einfach wie möglich ist, wird dem Benutzer auf dem OLED-Display ein QR-Code angezeigt, welchen er nur noch scannen muss, um die Weboberfläche zu öffnen. Auch hier darf der Eintrag im Crontab nicht fehlen:
\begin{verbatim}
    @reboot ~/klimatrack/display_qr_web_api.py
\end{verbatim}
\section{Abschluss}
Nach einem Neustart sollten nun alle Komponenten laufen und zur Verfügung stehen.
\end{document}
